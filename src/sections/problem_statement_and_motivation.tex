\begin{figure}[ht]
    \centering
    \sbox0{\includegraphics[width=10cm]{human_assisted_landing}}
    \begin{minipage}{\wd0}
      \usebox0
      \caption{Non-ideal, human-assisted landing at the Fagradalsfjall volcano in the absence of an autonomous, safe landing method that considers the surrounding environment.}
      \label{figure:hand_landing}
    \end{minipage}
\end{figure}

\section{Problem Statement and Motivation}

The goal of the proposed research is to explore the topic of
autonomous, unstructured drone landing.
Current autonomous landing methods have at least one of the following disadvantages:
they are blind to obstacles,
they require \textit{known} landing sites,
or they depend on sophisticated ground control stations for offloading of expensive computation.
The proposed research targets a gap in current autonomous landing methods.
Specifically, we aim to develop a method for quickly analyzing terrain
and identifying safe landing sites using only embedded computational hardware
and a minimal set of sensors.

Landing is a particularly difficult aspect of drone flight,
owing mainly to its risky nature and required precision.
As a result, most drone landings are carried out by a human operator,
inherently limiting the applicability of autonomous drones.
Some autopilot software includes an API for \textit{precision landing},
which allows a drone to localize and direct itself with respect to a landing pad during an autonomous landing,
according to data provided by external sensors and programs.
However, there is no particular method of autonomous landing in widespread use.
As autonomous and semi-autonomous drones are not able to reliably handle landings
on rough terrain or in non-ideal conditions, human operators often disable
autonomous control during landing (opting for full manual control),
or abuse/hack the landing system by descending to a low altitude,
grabbing hold of the drone,
and disabling the motors,
as shown in Figure~\ref{figure:hand_landing}.
Aside from potentially exposing users to dangerous rotors,
this landing technique showcases the limitations induced by a lack of
autonomous landing method.

In sufficiently flat, large areas, fully autonomous drone missions can end with a GPS-based
autonomous landing which is blind to obstacles in the environment.
However,
intuitively and demonstrably,
this can lead to crash-landings at landing sites that have obstacles within
the error radius of the GPS,
which can be anywhere from a few centimeters to a few meters.
In the available open source autopilot softwares,
obstacles are simply not handled,
and drones will continue their landing attempts even if fatally obstructed.