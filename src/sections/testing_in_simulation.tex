\section{Testing in Simulation (1 page)}

As the terrain classifiers will output a mask, we will have to post-process the data
to make it compatible with an autonomous drone system.
We will therefore create an output wrapper
to find contiguous regions of high predicted safety (i.e.~estimated safety above
some experimentally-determined threshold),
and to track these regions over time.
We will then select a region for landing, and create commands to direct the drone
towards that region in a similar manner to the method outlined in Section \ref{section:proof_of_concept}.
We will make no explicit estimate of the region's position relative to the drone in terms
of real world distance,
as this is unnecessary and requires more sophisticated wrappers which take into account the distortion
parameters of the camera in order to associate pixels with physical points (as in the case of fiducial markers).
It is simpler and more generalizable to different platforms to choose a safe landing region,
calculate its centroid, normalize its coordinates to the interval $[-1,1]$
based on the pixel dimensions of the camera/mask,
and then convert these values to Virtual Stick inputs (in the case of a DJI flight controller),
or to velocity setpoints (in the case of ArduPilot or PX4 in simulation).

After creating the output wrapper, we will move to testing the networks in simulated environments
in AirSim.
We will use multiple environments - both those from which the training data originates,
and also unfamiliar environments - to get an idea of the generalizability of the networks.
These networks will be evaluated on the number of successful landings they execute in a number
of different scenarios.
