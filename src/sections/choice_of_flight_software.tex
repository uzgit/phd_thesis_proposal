\section{Choice of Flight Software/Hardware and Drone Upgrades}
\label{section:choice_of_flight_software}

There are two general stages of testing in the context of drone control algorithms:
testing in simulation and testing in the real world.
Testing in simulation is an intuitive first step in testing,
as it allows to test control algorithms without added logistical, time, or financial cost.
It also provides an idealized world, free of complicating factors such as wind, battery life, noisy sensors, etc.,
isolating the control algorithm as the only testing variable.
However, simulation does not prove that a system works in real life.
Therefore, after control algorithms have been proven to work in theory (simulation),
they must prove themselves again in the real world, with all the implied complexities.

ArduPilot and PX4 are key software packages that can be used for testing in simulation.
Previous work in this project and in others have proven worthiness of ArduPilot/PX4 and Gazebo in terms of simulated autonomous control.
They have complete integration with both simulation environments that are of interest in the upcoming research:
Gazebo and AirSim.
Both softwares have APIs in Python, C++, and ROS,
and both offer positional/velocity control in several coordinate frames.
They also offer methods of controlling external actuators (e.g.~servos, gimbals, etc) programmatically.
This means that both systems are technically able to accomplish all the necessary tasks.
Since they are free, open source, and easy to integrate into other important software, they are ideal for testing in simulation.
In the real world, however, these autopilots exhibit quirks that are seemingly a lack of quality control
stemming from their open source nature.
For example, the \texttt{vmount} command of PX4, which controls a gimbal, can sometimes actuate unpredictably
in the \textit{pan} axis when given only a command in the \textit{tilt} axis from the Python API PyMAVLink.
Both PX4 and ArduPilot have trouble maintaining stable autonomous flight in Iceland when using only conventional GPS modules,
since GDOP is often high in Iceland (meaning that GPS precision is low).
Attempting to fly in autonomous mode in this case can result in ``fly-offs'' where the drone flies away uncontrollably.
Moreover, since there is a whole world of sensors that can interface with both ArduPilot and PX4,
there is also a whole world of meandering setup procedures (with questionable documentation) and potential problems with each of them,
and with each of their combinations.
For example, we have already bought a Cube Orange flight controller (supported by both ArduPilot and PX4) to replace the
Wookong flight controller on the heavy-lift drone mentioned in Section~\ref{section:ir_drone}.
We have planned to run PX4 on it, and use the Here3 RTK GPS system via its CAN interface.
However, according to the setup instructions~\cite{here3_manual},
we must first install ArduPilot solely to set a CAN ID for the GPS, before re-installing PX4.
In a worse case, a LIDAR rangefinder (used to accurately detect drone altitude) that was thought to be compatible with
ArduPilot, was eventually found and documented to have unpredictable 13-meter jumps in detected altitude~\cite{lidar_lite_rangefinder}:
\begin{quote}
    When the problem happens the 13m offset usually locks in place,
    so all remaining readings from the Lidar for the rest of the flight will give a 13m offset.
    There have however been cases where the 13m offset disappears after a few seconds or minutes.
\end{quote}
This issue is indicative of many issues within the world of ArduPilot and PX4.
While it is not a prohibitive issue that a single sensor might be learned to have serious problems with a particular flight control software/setup,
it adds to both the uncertainty and overhead in performance verification when creating drone systems.
Finally, there is a set of ground control stations (software to provide an interface for monitoring and controlling a drone)
for ArduPilot and PX4, consisting mostly of Mission Planner~\cite{mission_planner} and QGroundControl~\cite{qgroundcontrol}.
These are free and generally perform their necessary tasks, but are well-known to be glitchy
and incompatible with certain autopilot versions.
In short, drone systems based on ArduPilot or PX4 require a slow testing phase of their basic functionality
before they can be considered reliable enough for testing autonomous navigation techniques such as those in the focus of this research.
On the other hand, there is a large user base that is consistently exploring the space of available system setups with this software,
meaning that issues can often be detected and communicated relatively quickly.

DJI flight control software (and corresponding hardware) is more suited to real world testing
because of its greater stability and reliability.
ArduPilot and PX4 are open source, so while they lack the profit motive and black-box nature of DJI,
they also unfortunately lack its stability.
The most crucial result of this is the inability of such systems to enter an autonomous flight mode
when GPS precision is low (i.e.~GDOP is high) which is the case nearly always in Iceland.
Potential solutions include using RTK systems, but these require base stations, increasing the hardware complexity
of the required system.
They also require that the base station be enabled for some time before increasing GPS precision.
By contrast, many DJI drones (those with external distance/motion sensors) are stable in completely GPS-denied environments,
and allow for autonomous control even in those circumstances.
Additionally, DJI flight controllers can function in GPS mode with a higher GDOP,
such that they allow autonomous flight with lower GPS quality, albeit with minor drift/inaccuracy.
Moreover, DJI flight controllers function well as plug-and-play items,
requiring minimal setup while offering good stability and reliability.
While DJI's API is more closed, it is well-documented and does not exhibit unexpected behavior in our experiments.
Autonomous control of the drone and gimbal is available primarily through virtual stick inputs,
wherein the companion board requests virtual control from the flight controller,
and then can control the drone and gimbal by simulating user control inputs,
which can correspond either to target velocities or positions.
This allows for more intuitive autonomous control with high level checks, such as maintenance of obstacle avoidance.
In terms of disadvantages, drones without a companion board \textit{must} interface via the Mobile SDK through either an iOS or Android app,
which implies many constraints on methods of connection and transmission of data.
For example, in the case of the autonomously landing DJI Spark, video from the drone needed to be compressed on the \textit{tablet}
in an Android app, before wireless transmission to the companion board.
However, in the case of custom drones which do include companion boards, the connection infrastructure and programming setup are more flexible and less constraining.
DJI also provides several Android and iOS apps that allow for stable, reliable control of their drones.
There are also several free and open source code samples available from DJI so that users can create their own
such applications, as we have done in Section~\ref{section:proof_of_concept}.
However, DJI systems are often vastly more expensive than their open source counterparts,
they present logistical challenges in getting them to Iceland,
and their proprietary, black-box nature means that they can be unpredictably discontinued or their functionality can be prohibitively changed.

Overall, ArduPilot and PX4 are perfect options in the case of simulation,
since they are free and provide reliable control in idealized simulation environments.
However, control methods developed in simulation with ArduPilot or PX4 can be ported with relative ease to DJI flight controllers in the real world.
Although DJI has a proprietary simulator, it does not offer the realistic graphics and simulation that AirSim does,
and is not as versatile as Gazebo.
Therefore, we plan to use ArduPilot and PX4 for the first phase of testing (in simulation).

We still have more testing and development to do in order to create reliable drones for real world testing,
and we will pursue both the open source and DJI routes.
We will test and improve our own drone systems to use the Cube Pilot Orange with RTK,
hoping that it will provide sufficient stability for autonomous testing.
We will also expand by using optical flow and LIDAR sensors on our custom drone sensors,
which should give more reliable position/velocity control.
We will change at least one of the hexacopter systems developed in the early part of this research to use either a
DJI N3 or A3.
These are the later versions of the Naza V2 and other DJI flight controllers which have been shown to provide stable
GPS-based control in Iceland.
The main difference between them and the current DJI flight controllers that we have is that the new ones support the
Onboard SDK, and therefore allow customized programmatic control, in a similar way to the DJI Spark.
However, we have enough room on our custom drone systems to embed significantly more hardware than the Spark has.
We will carry out the real world tests and data collection with the upgraded drone(s).