\section{Choice of Flight Software}
\label{section:choice_of_flight_software}

There are two general stages of testing in the context of drone control algorithms:
testing in simulation and testing in the real world.
Testing in simulation is an intuitive first step in testing,
as it allows to test control algorithms without added logistical, time, or financial cost.
It also provides an idealized world, free of complicating factors such as wind, battery life, noisy sensors, etc.,
isolating the control algorithm as the only testing variable.
However, simulation does not prove that a system works in real life.
Therefore, after control algorithms have been proven to work in theory (simulation),
they must prove themselves again in the real world, with all the implied complexities.

ArduPilot and PX4 are key software packages that can be used for testing in simulation.
Previous work in this project and in others have proven worthiness of PX4 and Gazebo in terms of simulated autonomous control.
They have complete integration with both simulation environments that are of interest in the upcoming research:
Gazebo and AirSim.
Both softwares have APIs in Python, C++, and ROS.
This research requires position control - and particularly body offset position control.
This means that the drone must follow a position setpoint that is given in the coordinate frame
that is centered on the drone's body and oriented with the same heading as the drone.
The position setpoint is given in terms of ENU (East, North, Up)
where East refers to the drone's right,
North refers to the drone's front,
and Up refers to the drone's top.
The setpoint is given in meters, so a position setpoint of (E, N, U) = (-2, 1, -1)
means that the drone should move 2 meters to the right,
1 meter forward, and 1 meter down.
In ArduPilot, this type of body offset position setpoint is technically supported,
but the firmware translates the position setpoint
directly to a GPS coordinate and relies on GPS for navigating to that place.
PX4 works also offers an API for generating body offset position setpoints.
in some cases the position setpoint must be translated into the NED (North, East, Down) frame,
by simply switching the North and East components, and negating the Up component to form the Down component.
In PX4, precision land setpoints are only supported in the coordinate frame centered on the drone's home (takeoff) position,
and oriented due north, and the coordinate frame is referred to as \texttt{LOCAL\_NED}.
For generating a body offset position setpoint, one therefore must add the current position and heading
to the target position and heading, but otherwise it works in a similar way to the position setpoints.
Both ArduPilot and PX4 also offers APIs for controlling a gimbal to aim cameras.
In ArduPilot this can be done with an RC override command, where RC input can be programmatically generated and then
forwarded to the output channels that control a gimbal,
as if the software is controlling the gimbal via switches/sliders on the controller.
PX4 provides a more sophisticated \texttt{vmount} interface that controls the gimbal with possible (user-configurable) consideration
of the orientation of the drone.

DJI flight control software (and corresponding hardware) are more suited to real world testing
because of its greater stability and reliability.
ArduPilot and PX4 are open source, so while they lack the profit motive and black-box nature of DJI,
they also unfortunately lack its robustness in terms of flight capability.
The most crucial result of this is the inability of such systems to enter an autonomous flight mode
when GPS precision is low (i.e.~GDOP is high) which is the case nearly always in Iceland.
Potential solutions include using RTK systems, but these require base stations, increasing the hardware complexity
of the required system.
They also require that the base station be enabled for some time before increasing GPS precision.
By contrast, many DJI drones (those with external distance sensors) are stable in completely GPS-denied environments,
and allow for autonomous control even in those circumstances.
Additionally, DJI flight controllers can function in GPS mode with a higher GDOP,
such that they allow autonomous flight with lower GPS quality, albeit with minor drift/inaccuracy.
In this case, drones with ArduPilot or PX4 can even be subject to fly-offs (complete loss of control).

In terms of their APIs that enable autonomous control, ArduPilot and PX4 offer more open interfaces
with implementations in C++, Python, and ROS, and DJI offers a more closed solution with fewer options.
However, the more open interfaces of ArduPilot and PX4 (via e.g.~MAVLink/PyMAVLink) are poorly documented
and have occasional quirky/unexpected behavior that requires extra overhead efforts to address.
For example, certain message/command attributes are sometimes ignored with no explanation in the documentation,
and explanations can only sometimes be found in user forums (and may require independent verification).
In some cases of aiming a gimbal at targets in PX4, there has even been destructive behavior
where a change in the commanded gimbal tilt resulted in an unexpected and uncontrollable change in gimbal pan,
with no explanation.
This implies more development time in the best case, and crashes in the worst case.
While DJI's API is more closed, it is better documented and does not exhibit such unexpected behavior in our experiments.
Autonomous control of the drone and gimbal is available primarily through virtual stick inputs,
wherein the companion board requests virtual control from the flight controller,
and then can control the drone and gimbal by simulating user control inputs,
which can correspond either to target velocities or positions.
This allows for more intuitive autonomous control with high level checks, such as maintenance of obstacle avoidance.
In terms of disadvantages, drones without a companion board \textit{must} interface via the Mobile SDK through either an iOS or Android app,
which implies many constraints on methods of connection and transmission of data.
For example, in the case of the autonomously landing DJI Spark, video from the drone needed to be compressed on the \textit{tablet}
in an Android app, before wireless transmission to the companion board.
However, in the case of custom drones which do include companion boards, the connection infrastructure and programming setup are more flexible and less constraining.

Overall, ArduPilot and PX4 are perfect options in the case of simulation,
since they are free and provide reliable control in idealized simulation environments,
but they do not perform as well as similar DJI products in the non-idealized environment of Iceland.
However, control methods developed in simulation with ArduPilot or PX4 can be ported with relative ease to DJI flight controllers in the real world.
Therefore, we plan to use ArduPilot and PX4 for the first phase of testing (in simulation),
and DJI products for real world testing.