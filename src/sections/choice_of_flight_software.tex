\section{Choice of Flight Software}
\label{section:choice_of_flight_software}

There are two general stages of testing in the context of drone control algorithms:
testing in simulation and testing in the real world.
Testing in simulation is an intuitive first step in testing,
as it allows to test control algorithms without added logistical, time, or financial cost.
It also provides an idealized world, free of complicating factors such as wind, battery life, noisy sensors, etc.,
isolating the control algorithm as the only testing variable.
However, simulation does not prove that a system works in real life.
Therefore, after control algorithms have been proven to work in theory (simulation),
they must prove themselves again in the real world, with all the implied complexities.

ArduPilot and PX4 are key software packages that can be used for testing in simulation.
Previous work in this project and in others have proven worthiness of PX4 and Gazebo in terms of simulated autonomous control.
They have complete integration with both simulation environments that are of interest in the upcoming research:
Gazebo and AirSim.
Both softwares have APIs in Python, C++, and ROS.
This research requires position control - and particularly body offset position control.
This means that the drone must follow a position setpoint that is given in the coordinate frame
that is centered on the drone's body and oriented with the same heading as the drone.
The position setpoint is given in terms of ENU (East, North, Up)
where East refers to the drone's right,
North refers to the drone's front,
and Up refers to the drone's top.
The setpoint is given in meters, so a position setpoint of (E, N, U) = (-2, 1, -1)
means that the drone should move 2 meters to the right,
1 meter forward, and 1 meter down.
In ArduPilot, this type of body offset position setpoint is technically supported,
but the firmware translates the position setpoint
directly to a GPS coordinate and relies on GPS for navigating to that place.
PX4 works also offers an API for generating body offset position setpoints.
in some cases the position setpoint must be translated into the NED (North, East, Down) frame,
by simply switching the North and East components, and negating the Up component to form the Down component.
In PX4, precision land setpoints are only supported in the coordinate frame centered on the drone's home (takeoff) position,
and oriented due north, and the coordinate frame is referred to as \texttt{LOCAL\_NED}.
For generating a body offset position setpoint, one therefore must add the current position and heading
to the target position and heading, but otherwise it works in a similar way to the position setpoints.
Both ArduPilot and PX4 also offers APIs for controlling a gimbal to aim cameras.
In ArduPilot this can be done with an RC override command, where RC input can be programmatically generated and then
forwarded to the output channels that control a gimbal.
PX4 provides a more sophisticated \texttt{vmount} interface that controls the gimbal with possible (user-configurable) consideration
of the orientation of the drone.

