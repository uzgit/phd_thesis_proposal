\section{Drone Upgrades}
\label{section:drone_upgrades}

The drone systems will need some modifications in order to accommodate the proposed unstructured landing systems in Iceland,
with the principal modification being the addition of secondary positional and velocity sensors.
This is motivated by the performance of DJI drones in GPS-denied environments and in environments with low GPS accuracy, i.e.~most of Iceland.
We will install and test optical flow sensors and LIDAR sensors, which give the drones the ability to
accurately estimate their velocity and altitude, respectively, without GPS.
This should give the drones a better ability to maintain their position without a human operator,
and therefore should allow them to operate in \textit{autonomous} mode.
Additionally, we will test a Pixhawk Cube Orange flight controller and Here 3 GPS (RTK compatible)
on one of the drone systems.
This will allow a more minimal setup inside the drone, since the flight controller will be self-contained,
and not a combination of Raspberry Pi and Navio2.
The GPS itself should be more accurate, and can be supplemented with an RTK base station if necessary.

We have several sensors for which we will design and build waterproof, protective cases,
and adapters such that they can be mounted in a gimbal:
\begin{itemize}[itemsep=-0.1cm]
    \item an Intel RealSense D435 RGBD camera
    \item an Intel RealSense D455 RGBD camera with integrated IMU
    \item an Intel RealSense L515 LIDAR module with integrated IMU
    \item a Texas Instruments IWR6843 60 GHz RADAR module
\end{itemize}
These sensors can provide real world analogues to the RGBD images and point clouds in the synthetic data set.
There will be minor additional overhead in interfacing these sensors with the companion boards,
e.g.~compiling their libraries and figuring out their dependencies with likely limited documentation.