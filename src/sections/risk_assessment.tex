\section{Risk Analysis}

There is little risk in the synthetic data set generation, as we have already generated similar
(albeit more basic) data sets in the past, and running AirSim does not pose any logistical risk or hazard.
There is slight risk in collecting the real world data sets, from the ever-present risk of damage
to equipment from crashes, to the difficulty in collecting data with high enough resolution and
operating within the motion/distance constraints of the sensors.
To mitigate this, we will run several missions to collect real world data over a long period of time,
so that our method of collecting and tagging the data can be refined.

We anticipate that the terrain classifier creation should pose little risk, in that it should be
feasible to create classifiers that give the desired results on the training data.
This is justified by the fact that the types of networks we will be training have been shown to
perform well in a variety of similar classification tasks.
However, there is the risk that the bulk of the training data, being synthetic, will not adequately
represent the real world.
This would mean that the terrain classifiers can perform well in simulation
but not in the real world.
We cannot mitigate this risk by only using real world data because of the time and effort
required to label the data.
We can attempt to mitigate it by generating synthetic data from simulated environments that are
similar to the environment where we will be testing - i.e.~by limiting the scope of the data
to environments similar to Iceland, instead of using e.g.~tropical environments,
or contrived environments that do not represent the real world at all.
Additionally, we can use human-made post-processing wrappers to handle undesired behavior -
e.g.~selecting conservatively from the safe landing regions that the network identifies.

Testing in simulation will have low risk, as AirSim already has plugins for both ArduPilot and PX4,
as well as drone models that can fly autonomously in the simulation.
This step will mostly consist of integrating existing components with the terrain classifiers and
pre-/post-processing wrappers.

The first part of testing on physical platforms (in a lab setting), poses low risk.
Creating the infrastructure to transfer data to an external board is not difficult,
and measuring the framerate and power consumption is also not difficult.
There is a risk in the case that none of the terrain classifiers can run fast enough,
or with sufficiently little power consumption, on our particular hardware platforms.
In this case we will need to consider getting different platforms, but at the very least we will
have collected relevant and helpful data on the computational requirements of the terrain classifiers.
In the case that we both cannot generate networks that run fast enough on our hardware,
and we simultaneously cannot obtain sufficiently fast platforms,
we can still use sophisticated ground-based hardware for real time performance
(similar to the system architecture in the proof of concept),
and make incremental improvements in efficiency.
However, in this case we will have to give up the goal of running our solutions purely onboard the drone.

Testing the system for real world landings is the most risky part of the research,
not just in terms of logistical considerations and potential of crashes,
but also in that it depends on the high performance of the systems developed in the previous steps,
and also on their integration together in a single system.
The biggest risk that we anticipate is that the training data does not correspond well enough to the
real world for the system to properly identify safe landing sites.
Although we will try several ways to mitigate this risk, it cannot be entirely eliminated.
However, in the case that the synthetic data do not transfer to the real world,
we have a two-pronged basis on which to move forward.
First, we will have already conducted research on the types of network architectures,
sensor data, and pre-/post-processing wrappers that perform well in simulation,
and we will still be able to use these.
Second, we will have physical drone platforms that we can use to collect more real-world data,
and we can label this data according to simple, slow, human-made methods,
without considering segmentation masks.
Then, we can re-train the networks that we have on purely real-world data.